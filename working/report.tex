\subsection{Gas Optimization}\label{gas-optimization}

\subsubsection{\texorpdfstring{\texttt{batchRefund()} logic can be
simplified}{batchRefund() logic can be simplified}}\label{batchrefund-logic-can-be-simplified}

\textbf{Severity:} Gas Optimization

\textbf{Context:} \emph{(No context files were provided by the
reviewer)}

\textbf{Description:} The \texttt{batchRefund()} function emits an event

\begin{minted}[]{solidity}
BatchRefundProcessed(uint256 usersCount, uint256 totalAmount);
\end{minted}

and makes use of the following logic to track \texttt{usersCount} and
\texttt{totalAmount} refunded in the batch:

\begin{minted}[]{solidity}
uint256 batchTotalAmount = 0;
uint256 successfulRefunds = 0; // for tracking user count

for (uint256 i = 0; i < usersCount; i++) {
    _processSingleRefund(users[i], amounts[i], merkleProofs[i]);
    batchTotalAmount += amounts[i];
    successfulRefunds++;
}
\end{minted}

But this local variable \texttt{successfulRefunds} is not required. The
\texttt{batchRefund} only ever executes in full, it succeeds only if all
refunds succeed as per the input, and we already have the user count
from the input \texttt{users} array.

\textbf{Recommendation:} Consider using the already known users count
for event data, and removing the \texttt{successfulRefunds} variable.

\subsection{Informational}\label{informational}

\subsubsection{Incorrect natspec}\label{incorrect-natspec}

\textbf{Severity:} Informational

\textbf{Context:}
\href{https://cantina.xyz/code/c2523b52-5902-4b5e-9c3e-cd4ad285c076/src/MegaSaleRefund.sol\#L160}{MegaSaleRefund.sol\#L160}

\textbf{Description:} The natspec documentation above
\texttt{\_processSingleRefund()} function says:

\begin{minted}[]{solidity}
// PREREQUISITE: sale contract must grant TOKEN_RECOVERER_ROLE to this contract
// @dev This is checked in _processRefund() via sale.hasRole(sale.TOKEN_RECOVERER_ROLE(), address(this))
\end{minted}

But this is an incorrect description. The mentioned check is actually
done in \texttt{refund()} and \texttt{batchRefund()} functions, and
\texttt{\_processRefund()} function does not even exist.

\textbf{Recommendation:} Consider changing the natspec comments.

\subsubsection{\texorpdfstring{Make sure that \texttt{TokenSale} entity
addresses can handle refunded
USDT}{Make sure that TokenSale entity addresses can handle refunded USDT}}\label{make-sure-that-tokensale-entity-addresses-can-handle-refunded-usdt}

\textbf{Severity:} Informational

\textbf{Context:} \emph{(No context files were provided by the
reviewer)}

\textbf{Description:} The \texttt{TokenSale} contract provided a way for
users to bid with USDT, and the \texttt{MegaSaleRefund} contract is
aimed at providing 10\% rebates for users who opted in to lock the
\$MEGA tokens for a year.

The \texttt{refund()} function here uses the input \texttt{to\ address}
to check user entry in the merkle tree (list of all user entities that
are eligible for refund), and also sends the refunded USDT back to this
"to" address. The user does not have a way to receive the refunds at a
different address.

This works fine if all entities that interacted with \texttt{TokenSale}
contract were EOAs. But if they were contracts with hardcoded pathways,
they might not be able to pull those USDT out after receiving the refund
at their own contract.

\textbf{Recommendation:} Consider checking if user entities would have a
problem handling received USDT refunds, if they used smart contracts for
the public sale.
